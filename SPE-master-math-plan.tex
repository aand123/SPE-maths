\documentclass[10pt]{article}
%\usepackage{mycommands}
\usepackage{mathrsfs,footnote}
\pagestyle{empty}
\begin{document}
\section*{Mathematics for Software Engineering} 
\subsection*{Preliminary Lecture Plan and some\\
  Suggestions for Computer Laboratory Work}

\vspace{3ex}
\savenotes
\begin{tabular}{|p{0.08\textwidth}|p{0.3\textwidth}|p{0.15\textwidth}|
    p{0.5\textwidth}|}
  \hline
  Week&Lectures&Grimaldi&Lab work (suggestions)\\
  \hline % Week 1:
  1 & Introduction & &
  \\
  & Generating functions % and and recurrence relations
  & 9 & Combinatorics via convolutions
  \\
  & Algorithm complexity, the $\mathcal{O}$ and NP-concepts & 5.7,8 &
  Implement, clock and assess order complexity of sorting algorithms
  \\\hline % Week 2:
  2 & Recurrence, Z-transf & 10 & Alg.~complexity through recurrence
  \\
  \hline
  3&Abstract algebra&14,16.1--3&\\
  %&(Cryptology)&&(RSA Encryption\footnotemark[2])\\
  &Elem of coding theory&16.6--& Coding examples \\
  \hline
  4&Automata, formal machines&6,7&Implementing formal machines\\
  &Algorithmic graph theory\footnote{Partly covered in discrete
    mathematics course}&11,12&\\
  \hline
  5&Optimization in networks, shortest path\footnote{Covered in
    discrete mathematics course}, minimal spanning tree\footnotemark[2],
  maximum flow&13&Optimization in networks\\
  \hline
  6&Other optimization algorithms&
  &Transportation simplex method\\
  &Optimization in spreadsheets&&\\
  \hline
\end{tabular}
\spewnotes
\vspace{3ex}

\subsubsection*{Lectures}
The lectures cover mathematical concepts, theories and methods that
are relevant for the course. Understanding of the lecture material is
necessary for the labs and examination. The lectures may demonstrate
the use of tools for problem-solving in a mathematical context.

\subsubsection*{Labs}
The labs are designed as practical, hands-on exercises in solving
problems with the help of mathematical methods. Each lab gives a
practical problem that has to be analysed and formalised using the
relevant methods introduced in the lectures. When necessary additional
understanding of the theory should be gained by reading books and
other material. After that a solution should be produced and
implemented in Python as a script. Some labs may include
experimentation on alternative solutions to the problem with a
following writing of a short scientific report on the results.

\subsubsection*{Examination}
The examination includes questions concerning the mathematical methods
that are to be answered in writing. The questions are based on the
lectures and course literature.


\end{document}
