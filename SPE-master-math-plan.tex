\documentclass[10pt]{article}
%\usepackage{mycommands}
\usepackage{mathrsfs,footnote}
\pagestyle{empty}

\voffset-0.7in
\hoffset-0.7in
\textwidth=35pc\textheight=734pt\parindent=0pt

\begin{document}
\section*{Mathematics for Software Engineers}
\subsection*{Preliminary Syllabus}
% Lecture Plan and some\\ Suggestions for Computer Laboratory Work}

\vspace{3ex}
\savenotes
\begin{tabular}{|c|p{0.35\textwidth}|p{0.10\textwidth}|
    p{0.4\textwidth}|}
  \hline
  Week&Lectures&Grimaldi&Lab work (suggestions)\\
  \hline % Week 1:
  1 & Course presentation/  & & \\
    & Introduction to Python & &
  \\
  & Generating functions
  & 9 & Combinatorics via convolutions
  \\ \hline
  & Algorithm complexity \hfill\break
    the $\mathcal{O}$ and NP-concepts & 5.7, 8 &
  Implement, clock and assess order of complexity of sorting algorithms
  \\\hline % Week 2:
  2 & Recurrence relations & 10 & \\
    & The Z-transform      &    & Alg.~complexity through recurrence
  \\
  \hline
  3 & Abstract algebra&14, 16.1-3&\\
    & Elements of coding theory&16.6-9& Coding examples \\
  \hline
  4& Cryptology & 16.4 &RSA encryption\footnotemark[2]\\
   &Automata, formal machines&6, 7&Implementing formal machines\\
  & Overview: Graph theory\footnote{Partly covered in discrete
    mathematics course} & 11, 12 & \\
  \hline
  5& Optimization in networks,\hfill\break
    Shortest path\footnote{Covered in discrete mathematics course},\hfill\break
    Minimal spanning tree\footnotemark[2],\hfill\break
    Maximum flow&13&Optimization in networks\\
  \hline
  6&Other optimization algorithms&
  &Transportation simplex method\\
  &Optimization in spreadsheets&&\\
  \hline
\end{tabular}
\spewnotes
\vspace{3ex}

\subsubsection*{Course book}

Ralph P. Grimaldi,
\emph{Discrete and Combinatorial Mathematics},
Pearson, ISBN: 9780321211033.


\subsubsection*{Lectures}
The lectures cover mathematical concepts, theories and methods that
are relevant for the course. Understanding of the lecture material is
necessary for the labs and examination. The lectures may demonstrate
the use of tools for problem-solving in a mathematical context.

\subsubsection*{Labs}
The labs are designed as practical, hands-on exercises in solving
problems with the help of mathematical methods. Each lab gives a
practical problem that has to be analysed and formalised using the
relevant methods introduced in the lectures. When necessary additional
understanding of the theory should be gained by reading books and
other material. After that a solution should be produced and
implemented in Python as a script. Some labs may include
experimentation on alternative solutions to the problem with a
following writing of a short scientific report on the results.

\subsubsection*{Examination}
The examination includes questions concerning the mathematical methods
that are to be answered in writing. The questions are based on the
lectures and course literature.


\end{document}
